% v2-acmsmall-sample.tex, dated March 6 2012
% This is a sample file for ACM small trim journals
%
% Compilation using 'acmsmall.cls' - version 1.3 (March 2012), Aptara Inc.
% (c) 2010 Association for Computing Machinery (ACM)
%
% Questions/Suggestions/Feedback should be addressed to => "acmtexsupport@aptaracorp.com".
% Users can also go through the FAQs available on the journal's submission webpage.
%
% Steps to compile: latex, bibtex, latex latex
%
% For tracking purposes => this is v1.3 - March 2012

\documentclass[prodmode,acmtecs]{acmsmall} % Aptara syntax

% Package to generate and customize Algorithm as per ACM style
\usepackage[ruled]{algorithm2e}
\renewcommand{\algorithmcfname}{ALGORITHM}
\SetAlFnt{\small}
\SetAlCapFnt{\small}
\SetAlCapNameFnt{\small}
\SetAlCapHSkip{0pt}
\IncMargin{-\parindent}

\usepackage{epstopdf}
\epstopdfsetup{update}
\usepackage{epstopdf-base}

%\epstopdfDeclareGraphicsRule{.eps}{pdf}{.pdf}{%
%  ps2pdf -sEPSCrop #1 \OutputFile
% }


% Metadata Information
\acmVolume{9}
\acmNumber{4}
\acmArticle{39}
\acmYear{2010}
\acmMonth{3}

% Copyright
\setcopyright{rightsretained}

% DOI
\doi{0000001.0000001}

%ISSN
\issn{1234-56789}

\makeatletter
\newcommand*{\rom}[1]{\expandafter\@slowromancap\romannumeral #1@}
\makeatother
\usepackage{tabu}
\newcolumntype{P}[1]{>{\centering\arraybackslash}p{#1}}
\newcommand{\RN}[1]{%
  \textup{\uppercase\expandafter{\romannumeral#1}}%
}

\usepackage{cite}

\usepackage[cmex10]{amsmath}

\usepackage{multirow}

\usepackage{array}

\usepackage{url}

\usepackage{subfig}

\usepackage{booktabs}

%\usepackage{caption}

\hyphenation{op-tical net-works semi-conduc-tor}

\newcommand{\EP}{EnergyPlus}
\newcommand{\DegC}{$^\circ$C}


% Document starts
\begin{document}


% Title portion
\title{Thermal-Sensor-Based Occupancy Detection For Smart Buildings Using Machine Learning Methods}
\author{QI HUA
\affil{Shanghai Jiao Tong University}
HENGYANG ZhAO
\affil{University of Californiat at Riverside}
HAIBAO CHEN
\affil{Shanghai Jiao Tong University}
YAOYAO YE
\affil{Shanghai Jiao Tong University}
Hai Wang
\affil{University of Electronic Science and Technology of China}
SHELDON X.-D. TAN
\affil{University of California, Riverside}
XIN LI
\affil{Carnegie Mellon University}
ESTEBAN TLELO-CUAUTLE
\affil{INAOE and CINVESTAV, Mexico}
}
% NOTE! Affiliations placed here should be for the institution where the
%       BULK of the research was done. If the author has gone to a new
%       institution, before publication, the (above) affiliation should NOT be changed.
%       The authors 'current' address may be given in the "Author's addresses:" block (below).
%       So for example, Mr. Abdelzaher, the bulk of the research was done at UIUC, and he is
%       currently affiliated with NASA.

\begin{abstract}
  In this article, we propose a novel method to detect the occupancy
  behavior of a building through the temperature and/or possible heat
  source information, which can be used for energy reduction and
  security monitoring for emerging smart buildings. Our work is based
  on a realistic building simulation program, \EP{}, from Department
  of Energy.  \EP{} can model the various time-series inputs to a
  building such as ambient temperature, heating, ventilation, and
  air-conditioning (HVAC) inputs, power consumption of electronic
  equipment, lighting and number of occupants in a room sampled in
  each hour and produce resulting temperature traces of zones (rooms).
Three machine learning based approaches for detecting human
  occupancy of a smart building are applied herein, namely: support
  vector regression (SVR) method, forward neural network (FNN) method
  and recurrent neural network (RNN). Experimental results with SVR
  method show that 4-feature model provides accurate detection rate
  giving a 0.638 average error and 0.0532 error ratio, and 5-feature
  model gives a 0.317 average error and 0.0264 error ratio. This
  indicates that SVR is a viable option for occupancy detection. FNN
  method show that 5-feature model manifests similar accuracy on
  occupancy detection. In RNN method, Elman's RNN (ELNN) can estimate
  occupancy information of each room of a building with high
  accuracy. It has local feedbacks in each layer and for a 5-zones
  building it is very accurate for occupancy behavior estimation. The
  error level, in terms of number of people can be as low as 0.0056 on
  average and 0.288 at maximum considering ambient, room temperatures
  and HVAC powers as detectable information. Without knowing HVAC
  powers, the estimation error can still be 0.044 on average, and only
  0.71\% estimated points have errors greater than 0.5.
\end{abstract}

\keywords{Smart building; support vector regression; neural network; indoor temperature; occupancy detection.}

\acmformat{Qi Hua, Hai-Bao Chen, Yao-Yao Ye, Sheldon X.-D. Tan, Xin Li, and Esteban Tlelo-Cuautle, 2016. Occupancy Detection in Smart Buildings Using Machine Learning Methods.}

% At a minimum you need to supply the author names, year and a title.
% IMPORTANT:
% Full first names whenever they are known, surname last, followed by a period.
% In the case of two authors, 'and' is placed between them.
% In the case of three or more authors, the serial comma is used, that is, all author names
% except the last one but including the penultimate author's name are followed by a comma,
% and then 'and' is placed before the final author's name.
% If only first and middle initials are known, then each initial
% is followed by a period and they are separated by a space.
% The remaining information (journal title, volume, article number, date, etc.) is 'auto-generated'.


\maketitle

\input intro.tex

\input energyplus_review.tex

\input machine_learning_review.tex

\input svm_nn_method_results.tex

\input rnn_method.tex

\input results.tex


\section{Conclusion}
A mathematical model based on SVR and neural network to detect
employee occupancy in a smart building, has been introduced. Three
machine learning based occupancy detection methods for smart buildings
through the thermal sensor temperature information and/or possible
heat source information, have been discussed. In all experiments, we
used the realistic building simulation program \EP{} to collect
training and validation datasets. Ambient factors, room temperature,
and/or HVAC power were selected as features to train Elman's recurrent
neural network (ELNN). In SVR model, two sets of features are offered
to feed off the model for different conveniences. The first set of
features is comprised of 4 features including solar factor, working
time, indoor temperature and outdoor temperature, which are regarded
as easily obtained features; whereas the second set of features adds
light energy as the fifth feature. In light of the experimental
results, 4-feature model has a quite accurate detection rate which
gives a 0.638 average error and 0.0532 error ratio. However, 5-feature
SVR model giving a 0.317 average error and 0.0264 error ratio has a
better performance than 4-feature model, which we consider as
moderating the under-fitting issue. This indicates that using SVR
model is a viable option when it comes to occupancy detection given
its convenience in data acquirement. In neural network, only 5-feature
model is established and it manifests similar accuracy on occupancy
detection.

In the recurrent neutral network based method, the resulting Elman
network can estimate occupancy information of each room of a building
with high accuracy. Using ambient factors and room temperatures only,
the average estimation error is 0.044, and only 0.71\% of the
estimated points have errors greater than 0.5 in terms of number of
people. This indicates that it is possible to precisely estimate the
occupancy only using ambient factors and room temperatures. With HVAC
powers added, the estimation can be even more accurate with even
simpler neural networks.

As a conclusion, what influences the model is the number of features
and the number of data points, virtually the two different methods SVR
and ELNN work similarly well for solving this problem. However,
further improvement is likely to occur when the model is refined to be
able self-improved using current data set in the future.



% Bibliography
\bibliographystyle{ACM-Reference-Format-Journals}
\bibliography{title2,./building_model_sim}
                             % Sample .bib file with references that match those in
                             % the 'Specifications Document (V1.5)' as well containing
                             % 'legacy' bibs and bibs with 'alternate codings'.
                             % Gerry Murray - March 2012

% History dates
%\received{February 2007}{March 2009}{June 2009}

% Electronic Appendix
%\elecappendix

\medskip



\end{document}
% End of v2-acmsmall-sample.tex (March 2012) - Gerry Murray, ACM


