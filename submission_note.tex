%\documentclass{article}
%\usepackage{enumitem}
%\begin{center}
%{\large Submission note}
%\end{center}
%\begin{document}




\section{Submission Note}


%\noindent
Some preliminary results of this article appeared in {\it IEEE
International Symposium on Circuits and Systems (ISCAS 2016)
}\cite{zhao2016learning}. In this journal submission, we have made
significant changes over the conference version. We believe the
difference is more than 40\% over the conference version of this
work. The details of changes are summarized below:
\begin{enumerate}
\item We further design and evaluate another learning-based technique,
    support vector regression (SVR), in addition to the recurrent neural network method (RNN)
    for thermal-sensor-based occupancy detection in smart buildings.  

\item Detail concepts of EnergyPlus and machine-learning based methods for occupancy 
detection have been added in Sections \ref{sec:energy_plus_review} and \ref{sec:machine_learning_review} so that the main content and the contribution of
  the new work can be better appreciated. Also, the details of feature selection 
and data configuration used in the two machine learning methods for occupancy detection have been added in Section \ref{sec:proposed_methods}. 

\item Experiment results of SVR based occupancy detection have been given in Section \ref{sec:results}.
Also, we added the discussion and comparison between the SVR and RNN methods in the experimental section.
    
\item We completely rewrote the abstract, introduction, problem
sections to reflect the new scope of the article. The content of the article, including the notations and figures,
have been substantially revised to improve the presentation.






  





%In the submission note part, can you please add more detailed notes showing the difference from the published ISCAS paper? The ISCAS paper is all about RNN with nothing to do with SVM/SVR, so the difference can focus on the SVR works
%
%\item First, we further design and evaluate another learning-based technique,
%    support vector regression, in addition to the neural network methods.
%
%\item (Please add more paragraphs showing additional works in detail. The
%    following are examples shown from submissions from previous topics)
%
%\item For solution quality study and analysis, we compared the both
%  optimization algorithms on a number of examples. We found the result
%  indicated that the previously proposed constrained optimization can
%  lead to near-optimal results in general.
%
%\item We showed the Pareto-like trade-off between the performance and
%lifetime. We also discussed the trade-off as long-term and short-term
%reliability effects.
%
%\item We completely rewrote the abstract, introduction, problem
%sections to reflect the new scope of the article.
%
%\item The content of the article, including the notations and figures,
%have been substantially revised to improve the presentation.


\end{enumerate}



%\end{document}
