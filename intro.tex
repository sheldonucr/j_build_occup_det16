\section{Introduction}
Building takes an instrumental role in energy consumption and
smartness of a building has a large impact on inhabitants. According
to statistics provided by US Department of Energy, 70\% of electricity
of all has been consumed by buildings every year. Recent efforts have
been poured into the awareness of improving efficiency in quite a few
facets, e.g., heating, ventilation, air conditioning (HVAC) system
\cite{10}\cite{12}, lighting\cite{8}, IT energy consumption management
within buildings\cite{1}\cite{2}, etc. Amongst the overall energy
usage of various aspects of buildings, the efficiency of HVAC systems
has a tremendous impact on energy consumption \cite{83}. On the
contrary, a few studies \cite{86} reveal that buildings utilizing
programmable thermostats virtually are more likely to consume more
energies than ones without using smart devices. Automatic thermostat
control systems have been developed in different approaches
\cite{87}\cite{88}, and plenty of techniques are applied in the course
of building the system.

Detecting the occupancy ( i.e. whether there are residents) in a
building or a room has applications ranging from energy reduction
to security monitoring.
% Smart building and Building Energy and Comfort Management systems
For instance, occupancy detection is critical for energy and comfort management
system in a smart building~\cite{Nguyen2013Energy}.
Using the occupancy information, HVAC and lighting can be
automatically controlled to reduce energy consumption while keeping
human comfort.

Due to the importance of detecting building occupancy, many methods
have been proposed in the past using different technologies such as
passive infrared sensors~\cite{Dodier2006Building}, wireless camera sensor
network \cite{Erickson2009Energy}, and applying sound level, case
temperature, carbon-dioxide (CO2) and motion to estimate occupancy
number~\cite{Ekwevugbe2013Real}.  Preheat \cite{8.10} built rooms with
active radio frequency identification (RFID) and sensors to detect
home occupancy. Mozer \cite{8.9} proposed a neural network method by
using the history data from embedded motion sensors and actives RFID
to explore occupancy rate. Thermostat \cite{8.11} also devoted a
similar approach through the employment of magnetic reed switches and
passive infrared sensors to take control of the HVAC system at
home. However, those methods are more expensive for deployment as
dedicated equipment is required.

Many works have given an approach under the circumstance that the
detection requires a comparatively strict requirements for sensors,
and obviously the requirements of sensors resulting in transformations
of infrastructures may dramatically increase expenditure when it comes
to the total cost of the building and system. Besides, overflow data
including a vast of different aspects of conditions with respect to
inhabitant data looms a latent possibility to burden inhabitants
psychological pressure, because a large number of people may not
prefer living under the supervision of a great deal of data that is
available to someone else. Moreover, it may lead to a threat of
leakage of personal data, thus threatening personal privacy.

% The approach we propose is capable of detecting the occupancy of
% employees given conditions of specific parameters of an office,
% including the building structure, wall material, infiltration
% coefficients, etc. We set up a vivid and real simulation of building
% in EnergyPlus to acquire the data set under certain conditions. In
% this model, sensors are not required to be highly precise or extremely
% accurate which are demanded in a number of previous models proposed by
% other approaches, and the model we proposed does not detect employee
% occupancy through precise detection of state-of-the-art sensors. The
% occupancy of employees is detected under statistic approach and
% historical data set, the result of which will not rapidly fluctuate
% when subtle ambient temperature changes occur.

% EnergyPlus is a whole building energy simulation program
% \cite{energyplus} that utilized by researchers, architects and
% engineers to model energy consumption, e.g., ventilation, lighting,
% heating, cooling, and plug and process loads, and water use in
% buildings with U.S. Department of Energy Building Technologies Office
% funding its development. EnergyPlus plays in the role as a modular and
% structured code which contains most popular features. It mainly is a
% functional engine with texts written in format of inputs and outputs.
% Via a heat balance engine, loads calculated is able to be modified by
% users in specific time step to simulate the building system. The
% EnergyPlus building systems simulation module which is with the
% variable time step is capable of calculating cooling and heating
% system and the response of electrical system. This system works more
% like an integrated simulation and it is playing an important role in
% calculations for plant sizing, system, occupant health calculations
% and occupant comfort while providing precise temperature detection.
% Moreover, integrated simulation also offers user authority to evaluate
% moisture adsorption and desorption, controls of realistic system in
% building elements, air flow in an interzone, and radiant cooling and
% heating systems. Data generated from EnergyPlus is most likely to be
% regarded as the authentic conditions of an office. In this model, a
% segment of data set is acquired from EnergyPlus, with which we combine
% real data to make detection on office occupancy.

% Due to the fact our approach is based on the existing office located
% in Chicago, hence it is highly related to the infrastructures of the
% office upon which we build our model including internal materials,
% internal loads, space conditioning, location, simulation period,
% ground temperatures, walls, floors, roofs, etc. Before we establish
% this model, we need to confirm some parameters of the office of a
% building. This model has a high demand for knowing the detail of a
% certain room or office, even the infrastructure of a building. We need
% the specific parameters of a room to establish a model and get access
% to a vast amount of data generated via EnergyPlus.

In this work, we propose a novel approach to detect occupancy under
specific conditions by applying machine learning methods, while it
does not require plenty of sensors to be installed in a certain
building where as it makes detection based on historical data. We
generate the mathematical model based on support vector regression
(SVR) method and neural network detect occupancy with two sets of
features in correspondence with different convenience. Feeding off the
features generated from EnergyPlus, the SVR model is able to yield
highly accurate results for occupancy detection which is a convenient
and new attempt in this field.

The model displays benefits of not requiring highly-precise data set
from sensors, therefore, it is able to reduce the equipment
expenditure in modern buildings given the fast development of smart
building, and two sets of features make it to be conveniently applied
under different circumstances. The experiments are performed by using
a realistic building simulation program, EnergyPlus, from Department of
Energy., which can model the various time-series inputs to a building
such as ambient temperature, heating, ventilation, and
air-conditioning inputs, power consumption of electronic equipment,
lighting and number of occupants in a room sampled in each hour and
produce resulting temperature traces of zones (rooms). The next
section describes EnergyPlus that is further applied to two machine
learning based methods: SVR and Elman's recurrent neural network (ELNN).